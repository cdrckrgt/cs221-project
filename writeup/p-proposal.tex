\documentclass{article}

% if you need to pass options to natbib, use, e.g.:
% \PassOptionsToPackage{numbers, compress}{natbib}
% before loading nips_2018

% ready for submission
% \usepackage{nips_2018}

% to compile a preprint version, e.g., for submission to arXiv, add
% add the [preprint] option:
\usepackage[preprint]{nips_2018}

% to compile a camera-ready version, add the [final] option, e.g.:
% \usepackage[final]{nips_2018}

% to avoid loading the natbib package, add option nonatbib:
% \usepackage[nonatbib]{nips_2018}

\usepackage[utf8]{inputenc} % allow utf-8 input
\usepackage[T1]{fontenc}    % use 8-bit T1 fonts
\usepackage{hyperref}       % hyperlinks
\usepackage{url}            % simple URL typesetting
\usepackage{booktabs}       % professional-quality tables
\usepackage{amsfonts}       % blackboard math symbols
\usepackage{nicefrac}       % compact symbols for 1/2, etc.
\usepackage{microtype}      % microtypography

\title{Transfer Learning with FlappyBird}

% The \author macro works with any number of authors. There are two
% commands used to separate the names and addresses of multiple
% authors: \And and \AND.
%
% Using \And between authors leaves it to LaTeX to determine where to
% break the lines. Using \AND forces a line break at that point. So,
% if LaTeX puts 3 of 4 authors names on the first line, and the last
% on the second line, try using \AND instead of \And before the third
% author name.

\author{
  Cedrick Argueta \\%\thanks{Use footnote for providing further
    % information about author (webpage, alternative
    % address)---\emph{not} for acknowledging funding agencies.} \\
  Department of Computer Science\\
  Stanford University\\
  Stanford, CA 94305 \\
  \texttt{cedrick@cs.stanford.edu} \\
  %% examples of more authors
  \And
  Austin Chow \\
  Department of Computer Science \\
  Stanford University\\
  Stanford, CA 94305 \\
  \texttt{archow@stanford.edu} \\
  \AND
  Cristian Lomeli\\
  Department of Computer Science \\
  Stanford University\\
  Stanford, CA 94305 \\
  \texttt{clomeli@stanford.edu} \\
}

\begin{document}
% \nipsfinalcopy is no longer used

\maketitle

\begin{abstract}

Reinforcement learning's growth in popularity in recent years is partly due to its ability to play some video games with a level of mastery that no human can reach. Transfer learning is popular in the field of deep learning, and using pre-trained models on certain tasks speeds up training time and increases performance significantly. In this project we aim to apply transfer learning to the popular video game \textit{FlappyBird} and analyze its performance compared to traditional reinforcement learning algorithms.
 
\end{abstract}

\section{Introduction and Scope}

\section{Approach}

\subsection{Method}

The goal of our project is twofold: we aim to evaluate deep reinforcement learning algorithms on the FlappyBird game, and also experiment with transfer learning and what impact it makes on the training process. 

\subsection{Expected Behavior}

Austin, give input output behavior

\subsection{Infrastructure}

The infrastructure for the game comes mostly from the PyGame Learning Environment and keras-rl packages. The PyGame Learning Environment provides a nicely wrapped implementation of FlappyBird, complete with sprites and the relevant game mechanics built in. Keras-rl provides a deep reinforcement learning framework that provides a simple interface for training agents. We take advantage of these two here, writing simple wrappers for the PLE game instance so that it is compatible with keras-rl.

The keras-rl package additionally provides an implementation of several algorithms that are applicable to FlappyBird, particularly Deep Q-networks and the SARSA algorithm. 
\subsection{Baseline and Oracle}

Cristian, define baseline and oracle

\section{Challenges}
\end{document}

